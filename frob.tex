\section{Eigenvalues of Frobenius on $l$-adic cohomology}
\subsection{Part I}
	Let $\alpha$ an eigenvalue of Frobenius acting on $H^i(V/\bar{k})$ where $H^*(V/\bar{k})$ is any one of the known cohomologies. Then by Poncar\'e duality $\frac{q^d}{\alpha}$ is an eigenvalue of Frobenius actiong on $H^{2d-i}(V/\bar{k})$. Thus, both $\alpha$ and $\frac{q^d}{\alpha}$ are algebraic integers. It follows that in the field $F= \mathbb{Q}(\alpha)$, $\alpha$ is a $\lambda$-adic unit for any prime $\lambda$ which does not lie over $p= \Char k$.
	
	Since, by the Riemann hypothesis, one has that every complex absolute value of $\alpha$ is $q^{i/2}$, one is left with the intriguing question: What are the $l$-adic absolute values of $\alpha$, for primes $l$ lying over over $p$?
	
	This is a question, which can by hindsight be seen to have had a long tradition --- being related to Stickelberger's determination of the $p$-adic nature of Gauss sums and the Chevalley-Warning theorem which says that if $N$ is the number of solution of a polynomial equation $\text{mod } p$.
	
	It is reasonable to expect thta the study of the above question should use a $p$-adic cohomology, and it dose. To describe, briefly, the know results, we introduce the \emph{Newton polygon} of a monotone increasing sequence $a_1, a_2, \cdots, a_m$ of non-negative rational numbers: by definition, it is the graph of the real-valued continuous piece-wise linear function $v_{a_1, a_2, \cdots a_m}$ on $[0,m]$ which takes the value $0$ at $0$ and whose derivate is the constant $a_j$ on the interval $(j-1,j)$.
		
	Now let $V/\mathcal{W}(k)$ be a smooth projective scheme. where $\mathcal{W}(k)$ is the ring of Witt vectors of $k$. Let $V/k$ be its reduction to $k$ through canonical morphism $ \spec (k) \to \spec(\mathcal{W}(k))$. Make the hypothesis that the de Rham cohomology of $V/k$ is extremely well-behaved: namely, $H^q(V/\mathcal{W}(k), \Omega^p)$ is a free $\mathcal{W}(k)$-module (whose rank we shall denote $h^{p,q}$-- the $(p,q)$-th Hodge number of $V/k$) for all $p,q$. Fix an integer $r$, and define the $r$-th Hodge polygon of $V/k$ to be the Newton polygon of the sequence of integers. 