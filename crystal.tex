\documentclass[11pt,a4paper]{scmsnotes}
\usepackage[utf8]{inputenc}
\usepackage[screen]{pdfscreen}
%\usepackage[all]{xy}
%\usepackage{marginnote}
\usepackage{notestemplate}
%\usepackage{fontspec}
%\setmainfont{Times New Roman}
\usepackage{tikz-cd}
%\usepackage{times}
%\usepackage{euler}
\title{Positive Characteristic Geometry}
\author{Chow.}
\begin{document}
\maketitle
%
\section{Witt Vector cohomology}
Let $p$ be a prime number. We then define the Witt polynomials ($\mathbb{Z}$-coefficient polynomials) respect to $p$ as follows
\[
\begin{aligned}
	W_0(x_0)&:= x_0 &\\
	W_1(x_0,x_1)&:= x_0^p + p x_1&\\
	\cdots&\\
	W_n(x_0, \cdots, x_n)&:= \sum_{i=0}^{n}p^i{x_i}^{p^{n-i}}
\end{aligned}
\]
Let $x=(x_0, \cdots, x_n), y= (y_0, \cdots, y_n)$. From Serre's theorem \cite{}, there are polynomials $S_0, \cdots, S_n$, $P_0, \cdots, P_n$ with $2n+2$ variables satisfying 
\[
\begin{aligned}
&W_n(x) + W_n(y) = W_n (S_0(x,y), \cdots, S_n(x,y))\\
&W_n(x) \cdot W_n(y)= W_n(P_0(x,y), \cdots, P_n(x,y))
\end{aligned}
\]
For example, 
\[
\begin{aligned}
&S_0(x,y)=x_0 + y_0 & & S_1(x,y)=x_1 + y_1 + \frac{x_0^p+y_0^p - (x_0 +y_0)^p}{p}&\\
&P_0(x,y)=x_0 y_0 & & P_1(x,y)= y_0^p x_1+ y_1 x_0^p + px_1 y_1
\end{aligned}
\]
Now, we define a new ring structure on $n$-couple for ring $R$, which is denoted by $\mathcal{W}_n(R)$. For all $X,Y in \mathcal{W}_n(R)$
\[
\begin{aligned}
&X + Y= (S_0(X,Y),\cdots, S_{n-1}(X,Y))\\
&X \cdot Y= (P_0(X,Y),\cdots, P_{n-1}(X,Y))
\end{aligned}
\]

Let $X$ be non-singular variety over perfect field $k$ of characteristic $p > 0$ with dimension $n$. $\mathcal{W}_k$ is sheaf of Witt vectors of lenght $k$.

We have three operators
\begin{itemize}
\item Frobenius $F \colon \mathcal{W}_k \sheaf{X} \to \mathcal{W}_k \sheaf{X}$ locally defined by $(a_0, \cdots , a_{k-1}) \mapsto (a^p_0, \cdots, a^p_{k-1})$;
\item Verschieburg $V \colon \mathcal{W}_k \sheaf{X} \to \mathcal{W}_{k+1} \sheaf{X}$ locally defined by $(a_0, \cdots, a_{k-1}) \mapsto (0, \cdots, a_{k-1})$;
\item Restriction $R \colon \mathcal{W}_{k+1}\sheaf{X} \to \mathcal{W}_k \sheaf{X}$ locally defined by $(a_0, \cdots, a_n) \mapsto (a_0, \cdots, a_{n-1})$
\end{itemize}
It is well-known that 
\[
RVF= RFV =FRV = p \cdot \id{\mathcal{W}_k \sheaf{X}}
\]
where the "$\cdot$" is multiplication in $\mathcal{W}_k \sheaf{X}$.

Let $\mathcal{W}(M):= \lim_{k} \mathcal{W}_{k+1}(M) \xlongrightarrow{R} \mathcal{W}_{k}(M)$ inverse limit of projective system induced by $R$. For example, since $W_k(\mathbb{F}_p) = \mathbb{Z}/p^k \mathbb{Z}$, we have 
$\mathcal{W}(\mathbb{F}_p) = \mathbb{Z}_p$ the $p$-adic integer ring. 

We have following two canonical exact sequence of $\sheaf{X}$-module
\begin{align}
	&0 \to \mathcal{W}_{k-1}\sheaf{X} \xlongrightarrow{V} \mathcal{W}_k \sheaf{X} \xrightarrow{R^{k-1}} \sheaf{X} \to 0\\
	&0 \to \sheaf{X} \to \mathcal{W}_{k}\sheaf{X} \xlongrightarrow{R} \mathcal{W}_{k-1} \sheaf{X} \to 0
\end{align} 
Witt vector cohomology is defined as 
\[
H^n(X,\mathcal{W}\sheaf{X})= \lim_{k} H^n(X,\mathcal{W}_k \sheaf{X})
\]
in projective system $[R:H^n(X,\mathcal{W}_{k+1}\sheaf{X}) \to H^n(X,\mathcal{W}_{k}\sheaf{X})]$.

\section{Varieties in positive characteristic under Frobenius}
Let $X$ be a scheme over $\mathbb{F}_p$. $F_X$ is called \textbf{absolute Frobenius morphism} of $X$ which is induced by $f \mapsto f^p$ on affine covers. Frobenius map $\sigma$ of field $\mathbb{F}_p$ induces endmorphism of $\spec \mathbb{F}_p$. It is also denoted by $F_p$. Furthermore, we have following base change

\begin{equation}
\begin{tikzcd}
X \arrow[rrd, "F_X", bend left] \arrow[rd, "F" description, dashed] \arrow[rdd, "f"', bend right] &  &  \\
& X' \arrow[d,"f'"] \arrow[r] & X \arrow[d, "f"] \\
& \spec \mathbb{F}_p \arrow[r, "F_p"] & \spec \mathbb{F}_p
\end{tikzcd}
\end{equation}
Since $F_X$ and $F_p$ are homeomorphisms on underlying topological space of spectrum $\spec \mathbb{F}_p$, $F$ is also homeomorphism on underlying space. Hence it is easy to see that $F_X$ is affine morphism of finite type. For all $\sheaf{X}$-module $\mathcal{F}$ on $X$, $(F_X)_*(\mathcal{F})$ is same to $\mathcal{F}$ as sheaves over abelian groups although the their $\sheaf{X}$-module structures are different. 
Let $B^i_X= \im (d: \Omega^{i-1}_X \to \Omega^{i}_X)$, then we have following exact sequences from definitions of Frobenius.
\begin{equation}
	\begin{tikzcd}
		 &\sheaf{X} \arrow[r, "F_X"] &\sheaf{X} \arrow[r,"d"] &B^1_X\\
	\end{tikzcd}
	\begin{tikzcd}
	&\sheaf{X'} \arrow[r,"F"]&F_*\sheaf{X} \arrow[r,"F_*d"]& F_* B^1_X\\
	\end{tikzcd}
\end{equation}
\begin{ex}
	Let $Y= \spec A , X = \mathbb{A}^n_Y$. We have
	\[ 
	\begin{aligned}
		\sheaf{X'}&= (A,F_A) \otimes_A A[x_1, x_2, \cdots,x_n] \\
				&\cong A^p[x_1,x_2,\cdots,x_n]
	\end{aligned}
	\]
\end{ex}
If $k$ is perfect field, i.e.\ Frobenius map $\sigma: k \to k$ is automorphism, then absolute Frobenius morphism is automorphism of scheme $\spec k$. Hence the pull-back $X' \to X$ is also isomorphism and under isomorphism relative Frobenius $F$ do same as absolute Frobenius $F_X$ on $X$. So we also denote $F_X$ by $F$ in this case without confusion.
\subsection{F-split}
\begin{secdefn}
	The $k$-scheme of characteristic $p>0$ is called Frobenius split if exact sequence from Cartier isomorphism
	\begin{equation}
		\begin{tikzcd}
		\label{exact1}
		& 0 \arrow[r]& \sheaf{X'} \arrow[r,"F^{\#}"] &F_* \sheaf{X} \arrow[r,"d"] & F_* B_X^1\arrow[r]& 0
		\end{tikzcd}
	\end{equation}
	is split exact.
\end{secdefn}

If $k$ is perfect, then we have absolute Cartier isomorphism $C^{-1}_{\text{abs}}$, which induce following exact sequences
\begin{align}
\label{exact2}&0 \to \sheaf{X} \to F_*\sheaf{X} \to F_*B_X^1 \to 0\\
&0 \to F_* B_X^n \to F_*\Omega^n_X \xlongrightarrow{C_{\text{abs}}} \Omega^n_{X} \to 0
\end{align}
Under assumption that $k$ is perfect, $X$ is Frobenius split if and only if \ref{exact2} is split. This is because in this condition, base change of $X$ along Frobenius morphism of field $k$ is isomorphic to $X$.
To simplify the notation, we denote $F_*B_X^i$ or $(F_X)_* B_X^i$ by $(B_X^i)$.

Suppose $\xi$ is corresponding extension class of \ref{exact1} in $\ext^1_{\sheaf{X'}}((B_X^1), \sheaf{X'})$. It follows that $X$ is $F$-split if and only if $\xi =0 $. In particular, if extension group $\ext^1_{\sheaf{X'}}((B_X^1), \sheaf{X'})$ vanishes, then \ref{exact1} must be split. Writing it in cohomology form, we have $H^1(X', (B_X^1)^{\vee}) = \ext^1_{\sheaf{X'}}((B_X^1), \sheaf{X'})$.

\begin{seclemma}\label{lemma1}
	If $X$ is $F$-split, then $H^i(X,B_X^1)=0$ for all $i\geq0$.
\end{seclemma}
\begin{proof}
	Exact sequence \ref{exact2} being splitting implies it induces isomorphism 
	\[
	\sheaf{X} \oplus (B_X^1) \cong (F_X)_*\sheaf{X}
	\]
	Hence at cohomolgy level, we have 
	\begin{equation}
		H^i(X,\sheaf{X}) \oplus H^i(X,(B_X^1)) \cong H^i(X,(F_X)_* \sheaf{X})
	\end{equation}
	while $H^i(X,(F_X)_* \sheaf{X}) = H^i(X, \sheaf{X})$ for all $i\geq 0$, hence we have $\dim H^i(X,(B_X^1))=0$, so that $H^i(X,B_X^1)$ vanishes.
\end{proof}
\subsection{Slope Spectral Sequence}
Slope spectral sequence is constructed in works \cite{} of Luc Illusie on de Rham-Witt cohomology and comparasion to crystalline cohomology. It plays essential role in de Rham-Witt cohomology as Hodge-de Rham spectral sequence in de Rham cohomology. Actually, this spectral sequence gives explict descibtion of $F$-crystal structure on crystalline cohomology with de Rham-Witt cohomology. 

Let $X$ be smooth variety over perfect field $k$ with characteristic $p >0$. We have following de Rham-Witt complexes according to Illusie's works
\begin{equation}
\mathcal{W} \Omega^*_{X}:= [\mathcal{W}\sheaf{X} \to \mathcal{W}\Omega^1_X \to \cdots \to \mathcal{W} \Omega^n_X ]
\end{equation}
We have canonical filtration on $\mathcal{W} \Omega^*_X$ as Hodge filtration in Hodge theory such that
\begin{equation}
F^i \mathcal{W}\Omega^*_X = \mathcal{W} \Omega^{\geq i}= [0 \to \cdots \to 0 \to \mathcal{W}\Omega^i_X \cdots \to \mathcal{W} \Omega^n_X]
\end{equation}
Filtration ${F^i \mathcal{W}\Omega^* X }$ induces spectral sequence
\begin{equation}
E^{i,j}_1 \cong H^{j}(X, \mathcal{W} \Omega^i_X) \Rightarrow \mathbb{H}^n(X, \mathcal{W}\Omega^*_X)
\end{equation}
which is called \textbf{slope spectral sequence} of $X$.Illusie proved that hypercohomology $H^n(X, \mathcal{W} \Omega^*_X)$ computes crystalline cohomology $H^n_{\text{crys}}(X/W)$ and $E^{i,j}_1 \otimes K$ without $p$-torsion. Hence 
\begin{secthm}[Illusie]
If $H^j(X, \mathcal{W} \Omega^i_X)$ are all torsion-free, then the slope spectral sequence 
\begin{equation}
E^{i,j}_1 \cong H^{j}(X, \mathcal{W} \Omega^i_X) \Rightarrow \mathbb{H}^n(X, \mathcal{W}\Omega^*_X)
\end{equation}
degenerates in degree 1.
\end{secthm}
As name of this spectral sequence, it captures information of "slope" of $F$-crystal on $H^*_{\text{crys}}(X/W) \otimes K$. We have following corollary

\begin{seccor}[Bloch]
For any $i$, canonical homomorphism $H^*(X,\mathcal{W} \Omega^{\geq i}_X) \hookrightarrow H^*_{\text{crys}}(X/W)$ and $H^*_{\text{crys}}(X/W) \twoheadrightarrow H^*(X,\mathcal{W} \Omega^{\leq i}_X)$ induces isomorphism
\begin{align}
H^*(X,\mathcal{W} \Omega_X^{\geq i}) \otimes K &\xrightarrow{\sim} (H^* (X/W) \otimes K )_{\geq i}\\
(H^*_{\text{crys}} (X/W) \otimes K)_{[0,i[} &\xrightarrow{\sim} H^*(X, \mathcal{W} \Omega^{< i}_X) \otimes K  
\end{align}
\end{seccor}
\begin{proof}
We have following exact sequence 
\[
0 \to \mathcal{W} \Omega^{\geq i}_X \to \mathcal{W} \Omega^*_X \to \mathcal{W} \Omega^{\leq i-1}_X \to 0 
\]
Since connection map $0=d_1 \otimes K: H^{*-1}(X, \mathcal{W} \Omega^{\leq i-1}) \to H^*(X,\mathcal{W} \Omega^{\geq i}) $, upper exact sequence induces exact sequence of $F$-isocrystals
\[
0 \to H^*(X, \mathcal{W} \Omega^{\geq i}_X) \otimes K \to H^*_{\text{crys}}(X/W) \otimes K \to H^*(X, \mathcal{W} \Omega^{\leq i-1}_X)\otimes K \to 0
\]
We have already known that $F(\im H^*(X,\mathcal{W} \Omega^{\geq i}_X)) \subseteq p^i H^*_{\text{crys}}(X/W)$. Hence the slope of $H^*(X,\mathcal{W} \Omega^{\geq i}_X)$ is greater than $i$.
On other hand, 
\[
H^*(X, \mathcal{W} \Omega^{\leq i-1}_X) \cong  W_\sigma [[V]][F] \big/ (FV-p^{i},VF -p^{i})
\]
It implies that slope of $H^*(X,\mathcal{W} \Omega^{\leq i-1}_X)$ strictly less than $i$. Hence we can conclude the proof.  
\end{proof}

On other hand, we can describe Hodge polygon and Newton polygon explicitly. In fact, there is following theorem due to Ogus.
\begin{secthm}
	Let 
	\begin{itemize}
		\item $h^i = h^{i,n-i} = \dim_k H^{n-i}(X, \Omega^i_X)$
		\item $a_i = \dim_k H^{n-i}(X,\mathcal{W} \Omega_X^i) \big/VH^i(X,\mathcal{W} \Omega_X^i)$
		\item $a_i' = \dim_k H^{n-i}(X,\mathcal{W} \Omega_X^i)\big/FH^{n-i}(X,\mathcal{W} \Omega_X^i)$
	\end{itemize}
then
\[
\begin{aligned}
a'_{n}=0& &h^0 = a_0& &h^i= a'_{i-1}+a_i(1 \leq i \leq n) 
\end{aligned}
\]
and
\[
\begin{aligned}
& Hdg(t)= (\sum_{i\leq t}h^i, \sum_{i\leq t} i h^i)&\\
& Nwt(t)= Hdg(t) + (a'_{t-1}, ta'_{t-1})&
\end{aligned}
\]
\end{secthm}
\begin{seccor}
	\label{ordcor}
	If $H^j(X,\mathcal{W} \Omega_X^i)$ is torsion-free for all $i,j$, then $X$ being ordinary is equivalent to condition that $Nwt(t)= Hdg(t)$.
\end{seccor}
\begin{proof}
	TBA
\end{proof}
%%%%%%%%%%%%%%%%%%%%%%%%%%%%%%%%%%%%%%%%%%%%%%%%%
% !TEX root=crystal.tex
\subsection{Ordinary Varieties}
\begin{secdefn}
Let $X$ be a smooth, proper variety over field $k$ with $\Char k > 0$. We say $X$ is \textbf{ordinary} if it satisfies
\begin{align}
& &H^i(X,B_X^j)=0 & &\text{for all } i \geq 0, j > 0&
\end{align}
where $B^j_X = d(\Omega^{j-1}_{X/k})$ is the sheaf of boundaries of algebraic de Rham complex in degree $j$.
\end{secdefn}
Equivalently, $H^i(X',F_* B_X^i)=$ for all $i \geq 0,j>0$ implies  ordinarity of $X$ \footnote{\textcolor{red}{by Leray spectral sequence}}. where $F$ is relative Frobenius morphism $F=F_{X/k}: X \to X'$. 

More numerical description of ordinarity according to property \ref{ordcor}:

If $H^q(X,W\Omega_X^r)$ are all torsion free, then $X$ is ordinary if and only if for every $q$, Newton polygon defined by the slopes of action of Frobenius on $H^q_{\text{crys}}(X/W)$ is equal to Hodge polygon.

Here we list some properties of ordinary varieties.
\begin{secprop}
	Let $X$ be variety over perfect field $k$. If $X$ is ordinary and with trivial canonical bundle, then $X$ is splitting.
\end{secprop}
\begin{proof}
	Homological algebra tells us that exact sequence 
	\[
	0 \to \sheaf{X} \to F_* \sheaf{X} \to (B_X^1) \to 0
	\]
	lies in extension group $\ext^1((B_X^1),\sheaf{X})$ as extension class. Denote this class as $\xi$. If $\xi =0$ then this sequence splits. Hence if this extension group vanishes then it has no choice but splits. But we have $$H^1(X,(B_X^1)^{\vee}) = \ext^1((B_X^1),\sheaf{X})$$ and
	\[
	H^1(X,(B_X^1)^{\vee}) \cong H^{n-1}(X,B_X^1 \otimes \omega_X)^* \cong H^{n-1}(X, B_X^1)^*.
	\]Hence it actually vanishes if $X$ is ordinary.
\end{proof}

\begin{secprop}
\label{sec3}
	If $S$ is algebraic surface over positive characteristic perfect field, then $S$ being $F$-split $\Rightarrow$ ordinary.
\end{secprop}
\begin{proof}
	As Cartier isomorphism is actually trace map in Grothendieck duality, we have following two perfect pairings:
	\begin{align}
		\label{pair1}&F_* \sheaf{X} \otimes F_* \Omega_X^2 \to \Omega_X^2\\
		&F_* \Omega_X^1 \otimes F_*\Omega_X^1 \to \Omega_X^2
	\end{align}
	The pairing \ref{pair1} induces perfect pairings
	\begin{align}
	\label{pair2} (B_X^1) \otimes (B_X^2) \to \Omega_X^2& & \sheaf{X} \otimes \Omega_X^n \to \Omega_X^2
	\end{align}
	by interchange following exact sequences with exact functor $\hom(-, \Omega_X^2) = \hom (-, \sheaf{X})$.
	\[
	\begin{aligned}
	0 \to \sheaf{X} \to F_* \sheaf{X} \to (B_X^1) \to 0 \\
	0 \to (B_X^2) \to F_* \Omega_X^2 \to \Omega_X^2 \to 0
	\end{aligned}	
	\]
	Hence with lemma \ref{lemma1}
	\begin{equation}
		H^i(X, (B_X^2)) \cong H^{n-i}(X,(B_X^2)^{\vee} \otimes \Omega_X^2)^* \cong H^{n-i}(X, B^1_X)^*=0
	\end{equation}
	for all $0 \leq i \leq n$. This completes the proof.
\end{proof}
\begin{seccor}
	If $X$ is K3 surface, then $X$ being $F$-split $\Leftrightarrow$ ordinary.
\end{seccor}

Actually, we can genenralize for perfect pairings appearing in proof of \ref{sec3}. As in \cite{}, we use following notations
\begin{align}
	B_{m} \Omega^i_X := C^{-1}(B_{m-1} \Omega^i_X) & & B_1 \Omega^i_X :=(B_X^i) & & B_0 \Omega^i_X = \underline{0} \\
	Z_{m} \Omega^i_X := C^{-1}(Z_{m-1} \Omega^i_X) & & Z_1 \Omega^i_X :=(Z_X^i)& & Z_0 \Omega^i_X = \Omega^i_X
\end{align}
Note that we have following inclusion sequence
\begin{equation}
\underline{0}= B_0 \Omega^i_X \subset B_1 \Omega^i_X \subset \cdots \subset B_m \Omega^i_X \subset \cdots Z_m \Omega^i_X \subset Z_{m-1} \Omega^i_X \subset \cdots \subset Z_0 \Omega^i_X = \Omega^i_X
\end{equation}
$B_{m} \Omega^i_X $ and $Z_{m} \Omega^i_X$ can be viewed as locally free submodules of $F^m_*(\Omega^i_X)$.This implies that 
\begin{align}
&0 \to B_m \Omega^i_X \to F^m_* \Omega^i_X \to F_* ^m \Omega^i_X \big/ B_m \Omega^i_X \to 0 \\
\label{exact4}&0 \to Z_m \Omega^{n-i}_X \to F^m_* \Omega^{n-i}_X \to F^m_* \Omega^{n-i}_X \big/Z_m \Omega^{n-i}_X \to 0
\end{align}
are exact. $\Hom (-,\Omega^n_X)$ is exact since $\Omega^n_X$ is trivial. Applying it on \ref{exact4}, we get exact sequence 
\begin{equation}
0 \to \Hom (F^m_* \Omega^{n-i}_X\big/ Z_m \Omega^{n-i}_X , \Omega^n_X) \to \Hom(F^m_* \Omega_X^{n-i}, \Omega^n_X) \to \Hom(Z_m \Omega^{n-i}_X , \Omega^n_X) \to 0
\end{equation}
Since the middle term is isomorphism to $F_*^m \Omega^i_X$ induced from perfect pairing 
\begin{equation}
F_*^m \Omega^i_X \otimes F_*^m \Omega^{n-i}_X \to \Omega_X^n
\end{equation}
we can get following two canonical perfect pairings
\begin{align}
\label{pair3} B_m \Omega^i_X \otimes F_*^m \Omega_X^{n-i}\big/Z_m \Omega^{n-i}_X \to \Omega_X^n \\
Z_m \Omega_X^{n-i} \otimes F^m_* \Omega_X^i \big/ B_m \Omega^i_X \to \Omega^n_X
\end{align}
Note that we have canonical isomorphism $F^m_* \Omega^{n-i}\big/ Z_m \Omega^{n-i}_X \cong B_m \Omega_X^{n-i+1}$ induced by differential $d: \Omega_X^{n-i} \to \Omega_X^{n-i+1}$. Hence \ref{pair3} becomes $B_m \Omega_X^i \otimes B_m \Omega_X^{n-i+1} \to \Omega^n_X$.In particular, taking $i=1, n=2, m=1$ we get perfect pairing \ref{pair2}.

Notice that we have exact sequence
\[
0 \to B_1 \Omega_X^1 \to B_{m+1} \Omega_X^1 \xlongrightarrow{C} B_{m} \Omega_X^1 \to 0
\]
Hence we have surjective maps $C: H^n(X,B_{m+1} \Omega_X^1) \to H^n(X,B_m \Omega_X^1)$. So we get a projective system, then taking inverse limit
\[
H^n(X,B \Omega_X^1):=\lim_m H^n(X,B_m \Omega_X^1)
\]
\begin{secdefn}
	 The dimension of $H^n(X,B \Omega_X^1)$ is called $b$-number of $X$.
\end{secdefn}
In next subsection, we will use these to give explicit formula for height of Calabi-Yau varieties over positive characteristic which is also called $h$-number in \cite{}.

%%%%%%%%%%%%%%%%%%%%%%%%%%%%%%%%%%%%%%%%%%%%%%%%%
\subsection{Formal group law}
In algebraic geometry, \textbf{formal group} is referred by group object in category of formal schemes. If $\mathcal{G}$ is a functor from Artin algebras to groups which is left exact, then $\mathcal{G}$ is representable by formal group. $\mathcal{G}$ is called a \textbf{formal group law}.
Let 
\[
F_S^r(S)= \ker \left\{ H^r(X \times S, \mathbb{G}_m) \to H^r(X,\mathbb{G}_m)\right\}
\]
the cohomology is \'etale cohomology here. Theorem of Artin and Mazur say that this functor $F^r_S: \text{Art}_k \to \text{Ab}$ is pro-representable by formal group $\Phi_X$ if $X$ is proper Calabi-Yau with dimension $n$ over perfect field $k$. Moreover, we have that tangent space of $\Phi_X$ is isomorphic to $H^n(X, \sheaf{X})$. Hence since Calabi-Yau variety is with geometric genus 1, e.g. $\dim_k H^n(X, \sheaf{X})=1 $, $\Phi_X$ is formal group of dimension 1.  

%%%%%%%%%%%%%%%%%%%%%%%%%%%%%%%%%%%%%%%%%%%%%%%%%
\subsection{Height of Calabi-Yau varieties}
Let $X$ be Calabi-Yau variety over perfect field $k$ with characteristic $p$ and $\Phi_X$ be its formal Brauer group. Then we have following theorem 
\begin{secthm}\label{thm2.5}
If $X$ is of dimension $n$, then we have following formula for hegiht
\begin{equation}
h(\Phi_X) = \min \big\{ i \geq 1 : [F: H^n(W_i \sheaf{X}) \to H^n(W_i \sheaf{X})] \neq 0 \big\}
\end{equation}
\end{secthm}
Before giving the proof of this theorem, we will prepare some properties of Witt vector cohomology of Calabi-Yau varieties.
\begin{secprop}
	Let X be Calabi-Yau variety over perfect field $k$ of dimension $n$. We have
	\begin{enumerate}
		\item $H^i(X, \mathcal{W}_j\sheaf{X}) =0$ for all $j > 0, 1 \leq i \leq n-1$. Furthermore, $H^i(X, \mathcal{W}\sheaf{X})=0$;
		\item Pull-back of $R$ on cohomology $R: H^n(X,\mathcal{W}_k \sheaf{X}) \to H^n(X,\mathcal{W}_k \sheaf{X})$ is surjective with kernel $H^n(X,\sheaf{X})$;
		\item Assume that for some $0<k \leq n$ the map $F: H^k(X,\mathcal{W}_j \sheaf{X}) \to H^k(X,\mathcal{W}_j \sheaf{X})$ vanishes, then for all $0 \leq i \leq j$ the map $F \colon H^k(X,\mathcal{W}_i \sheaf{X}) \to H^k(X,\mathcal{W}_i \sheaf{X})$ vanishes too. Moreover, for all $H^n(X,\mathcal{W}_i \sheaf{X})$ is vector space over $K$;
		\item 
		\[
		0 \to H^n(X, \mathcal{W}_{i-1}\sheaf{X}) \xlongrightarrow{V} H^n(X,\mathcal{W}_i \sheaf{X})\xlongrightarrow{R^{n-1}} H^n(X,\sheaf{X}) \to 0
		\]
		and
		\[
		0 \to H^n(X,\mathcal{W} \sheaf{X}) \xlongrightarrow{V'} H^n(X,\mathcal{W}\sheaf{X}) \xlongrightarrow{R'} H^n(X,\sheaf{X}) \to 0
		\]
		are both exact.
	\end{enumerate}
\end{secprop}
\begin{proof}
	\begin{enumerate}
		\item Note that we have following exact sequence of sheaves for all $j \geq 1$
		\[
		0 \to \mathcal{W}_{j-1} \sheaf{X} \to \mathcal{W}_{j} \xlongrightarrow {R^{j-1}} \sheaf{X} \to 0
		\]
		Hence we have exact sequence of groups
		\[
		H^{i}(X, \mathcal{W}_{j-1} \sheaf{X}) \to H^{i}(X, \mathcal{W}_{j} \sheaf{X}) \to H^{i}(X,\sheaf{X})
		\]
		If $H^i(X,\mathcal{W}_{j-1})=0$ and $1 \leq i \leq n-1$, then $H^i(X. \mathcal{W}_j \sheaf{X}) =0 $ since two sides of this sequence are zero. Hence by induction, we complete the proof.
		\item We have following exact sequence 
		\[
		0 \simeq H^{n-1}(X, \mathcal{W}_{k-1}) \to H^n(X,\sheaf{X}) \to H^n(X,\mathcal{W}_k \sheaf{X}) \xlongrightarrow{R} H^n(X, \mathcal{W}_{k-1}\sheaf{X}) \to 0
		\]
		It implies $H^n(X,\sheaf{X}) = \ker \left(H^n(\mathcal{W}_k \sheaf{X}) \xlongrightarrow{R} H^n(X,\mathcal{W}_{k-1} \sheaf{X})\right)$.
		\item Note that following diagram commutes for all $j>i$
		\[
		\begin{tikzcd}
		H^n(X,\mathcal{W}_{j} \sheaf{X}) \arrow[r,"R^{j-i}",two heads] \arrow[d, "F"]& H^n(X,\mathcal{W}_i \sheaf{X}) \arrow[d, "F"]&\\
		H^n(X,\mathcal{W}_j \sheaf{X}) \arrow[r, "R^{j-i}",two heads] &H^n(X, \mathcal{W}_i \sheaf{X})&
		\end{tikzcd}
		\]
		Hence if left Frobenius map vanishes then so does right one.
	\end{enumerate}
\end{proof}

\begin{proof}[Proof of theorem \ref{thm2.5}]
	Let $h=h(\Phi_X)$. If $F$ is zero map on each $H^n(X,\mathcal{W}_{i} \sheaf{X})$, then $H^n(X,\mathcal{W}\sheaf{X})$ is $k$-vector space. But we have following exact sequence
	\[
	0 \to H^n(X,\mathcal{W}\sheaf{X}) \to H^n(X,\mathcal{W}\sheaf{X}) \to H^n(X,\sheaf{X}) \to 0
	\]
	and $H^n(X,\sheaf{X})$ is actually $k$-vector space. Hence $H^n(X,\mathcal{W} \sheaf{X})$ is not of finite dimension as $k$-vector space. Since $h= \rank _{W(k)} H^n(X,\mathcal{W}\sheaf{X})$, we can conclude that in this case $\Phi_X$ is of height $\infty$.
	
	Notice that 
	\[
	H^n(X,\mathcal{W} \sheaf{X}) \big/ VH^n(X,\mathcal{W}\sheaf{X}) \simeq H^n(X,\sheaf{X})
	\]
	as $W(k)$-module and $H^n(X,\sheaf{X})$ is naturally $k$-vector space, they are isomorphic as $k$-vector space. Hence from equality
	\[
	\rank_{W(k)}(D(\Phi_X)) = \dim_k H^n(X,\mathcal{W} \sheaf{X})\big/ FH^n(X,\mathcal{W} \sheaf{X}) + \dim_k H^n(X,\mathcal{W} \sheaf{X})\big/ V H^n(X,\mathcal{W} \sheaf{X})
	\]
	we get 
	\[
	\dim_k H^n(X,\mathcal{W} \sheaf{X})\big/ FH^n(X,\mathcal{W} \sheaf{X}) = h-1
	\]
	If $F:H^n(X,\mathcal{W}_i \sheaf{X}) \to H^n(X,\mathcal{W}_i \sheaf{X})$ is zero ,then projection
	\[
	[P_{i}]: H^n(X,\mathcal{W}\sheaf{X})\big/ FH^n(X,\mathcal{W} \sheaf{X}) \to H^n(X,\mathcal{W}_i \sheaf{X}) \big/FH^n(X,\mathcal{W}_i \sheaf{X})
	\]
	is well-defined and surjective since $R,F$ commutes. As vector space, $H^n(X,\mathcal{W}_i \sheaf{X})$ is of dimension $i$. This implies that we have inequality relation
	\[
	h \geq i+1
	\]
	Hence we get
	\[
	h \geq \min \big\{ i \geq 1 : [F: H^n(W_i \sheaf{X}) \to H^n(W_i \sheaf{X})] \neq 0 \big\}
	\]
	
	Conversely, it is sufficient to prove $FH^n(X,\mathcal{W}_{h-1} \sheaf{X}) =0$. 

	Manin's result in \cite{} implies that 
	\begin{equation}
		H^n(X,\mathcal{W} \sheaf{X}) \cong D(\Phi_X) \cong \frac{W(k)[F,V]}{(F - V^{h-1})} 
	\end{equation}
	Hence $FH^n(X,\mathcal{W} \sheaf{X}) \cong V^{h-1}H^n(X,\mathcal{W} \sheaf{X})$.
		\[
	\begin{tikzcd}
	V^{h-1} H^n(X,\mathcal{W} \sheaf{X}) \arrow[r] \arrow[d, two heads]& H^n(X,\mathcal{W} \sheaf{X}) \arrow[d,two heads]& \\
	0 =V^{h-1} H^n(X, \mathcal{W}_{h-1} \sheaf{X}) \arrow[r] & H^n(X,\mathcal{W}_{h-1} \sheaf{X})& 
	\end{tikzcd}
	\]
	commutes, this implies $FH^n(X ,\mathcal{W}_{h-1} \sheaf{X}) =0$. 
\end{proof}
We will call $h(\Phi_X)$ the $h$-number of $X$, denoted by $h(X)$.

\begin{seccor}
For proper Calabi-Yau variety $X$ over perfect field $k$, we have $b(X)= \dim_k H^n(X,\mathcal{W} \sheaf{X})/F H^n(X,\mathcal{W}\sheaf{X})$. Furthermore,
\[
\dim_k H^{n-1}(X,B_m \Omega_X^1)=\dim_k H^n(X,B_m \Omega_X^1) = \begin{cases}
\min \{m,h-1\}& \text{if } h < \infty\\
m& \text{if } h = \infty
\end{cases}
\]	
\end{seccor}

\begin{proof}
	We have exact sequence
	\[
	\mathcal{W}_m \sheaf{X} \xrightarrow{F} \mathcal{W}_m \sheaf{X} \xrightarrow{D_m} B_m \Omega_X^1 \to 0
	\]
	It induces 
	\[
	H^n(X,\mathcal{W}_m \sheaf{X}) \xrightarrow{F} H^n(X,\mathcal{W}_m \sheaf{X}) \to H^n(X,B_m \Omega_X^1) \to 0
	\]
	Taking limit with $m$, we get 
	\[
	H^n(X, \mathcal{W} \sheaf{X}) \xrightarrow{F} H^n(X,\mathcal{W} \sheaf{X}) \to H^n(X, B \Omega_X^1) \to 0
	\]
	exact. Hence $b(X)= h(X)-1$. 
\end{proof}

% TEX root=crystal.tex
\section{Eigenvalues of Frobenius on $l$-adic cohomology}
\subsection{Part I}
	Let $\alpha$ an eigenvalue of Frobenius acting on $H^i(V/\bar{k})$ where $H^*(V/\bar{k})$ is any one of the known cohomologies. Then by Poncar\'e duality $\frac{q^d}{\alpha}$ is an eigenvalue of Frobenius actiong on $H^{2d-i}(V/\bar{k})$. Thus, both $\alpha$ and $\frac{q^d}{\alpha}$ are algebraic integers. It follows that in the field $F= \mathbb{Q}(\alpha)$, $\alpha$ is a $\lambda$-adic unit for any prime $\lambda$ which does not lie over $p= \Char k$.
	
	Since, by the Riemann hypothesis, one has that every complex absolute value of $\alpha$ is $q^{i/2}$, one is left with the intriguing question: What are the $l$-adic absolute values of $\alpha$, for primes $l$ lying over over $p$?
	
	This is a question, which can by hindsight be seen to have had a long tradition --- being related to Stickelberger's determination of the $p$-adic nature of Gauss sums and the Chevalley-Warning theorem which says that if $N$ is the number of solution of a polynomial equation $\text{mod } p$.
	
	It is reasonable to expect thta the study of the above question should use a $p$-adic cohomology, and it dose. To describe, briefly, the know results, we introduce the \emph{Newton polygon} of a monotone increasing sequence $a_1, a_2, \cdots, a_m$ of non-negative rational numbers: by definition, it is the graph of the real-valued continuous piece-wise linear function $v_{a_1, a_2, \cdots a_m}$ on $[0,m]$ which takes the value $0$ at $0$ and whose derivate is the constant $a_j$ on the interval $(j-1,j)$.
		
	Now let $V/\mathcal{W}(k)$ be a smooth projective scheme. where $\mathcal{W}(k)$ is the ring of Witt vectors of $k$. Let $V/k$ be its reduction to $k$ through canonical morphism $ \spec (k) \to \spec(\mathcal{W}(k))$. Make the hypothesis that the de Rham cohomology of $V/k$ is extremely well-behaved: namely, $H^q(V/\mathcal{W}(k), \Omega^p)$ is a free $\mathcal{W}(k)$-module (whose rank we shall denote $h^{p,q}$-- the $(p,q)$-th Hodge number of $V/k$) for all $p,q$. Fix an integer $r$, and define the $r$-th Hodge polygon of $V/k$ to be the Newton polygon of the sequence of integers. 
	\[
	\underbrace{0,0,\cdots,0}_{h^{0,r}}, \ \ \underbrace{1,1,\cdots,1,}_{h^{1,r-1}}\ \ \underbrace{2,2,\cdots,2,}_{h^{2,r-2}},\ \ \cdots\ \ \underbrace{r,r,\cdots, r}_{h^{r,0}} 
	\]
	 Define the $r$-th Newton polygon of $V$ to be the Newton polygon of the sequence:
	 \[
	 \ord_q \alpha_1, \ord_q \alpha_2, \cdots, \ord_q \alpha_m\ \ (m= r\text{-th Betti number of  }V)
	 \]
	 where $\alpha_1, \cdots, \alpha_m \in \bar{K}$ are the complete set of eigenvalues (with multiplicities) of Frobenius acting on the $r$-dimensional $p$-adic cohomology of $V/k$; $\ord_q$ denotes the $p$-adic ord function normalized so that $\ord_q q =1$; and the $\alpha_j$ are indexed in such a manner that the above sequence is monotone increasing.
	 
	 What is known about the $r$-th Newton polygon of $V/k$ is the following:
	 
	 The $r$-th Newton polygon and the $r$-th Hodge polygon both end at the point $(m,rm/2)$ in the euclidean plane. The break-points (non-differentiable points) of these polygon occur at integral lattice points. The $r$-th Newton polygons lies in the closed region bounded by the $r$-th Hodge polygon, and the straight line joining $(0,0)$ and $(m,rm/2)$.
	 
	 Note: One has a further symmetry in the geometry of the Newton polygon implied by the strong Lefschetz theorem, and also the existence of algebraic cohomology classes in dimension $r$ implies that a part of the Newton polygon must have slope $r/2$.
	 
	 General remarks: It was an important discovery of Dwork, that (in a certain, unobvious, sense) the eigenvalues of Frobenius vary $p$-adic analytically if one varies the variety $V/k$ over a parameter space in characteristic $p$. A trace of this phenomenon may be seen in a theorem of Grothendieck: The Newton polygon rises under specialization of $V/k$.
	 
\subsection{Explicit description of Newton Polygon and Hodge Polygon}
Let $H^n = H^n_{\text{crys}}(X/W)$ be torsion free.  Let 
\begin{itemize}
	\item $h^i= h^{i,n-i}= \dim_k H^{n-i}(X,\Omega_X^i)$
	\item $a_i = \dim H^{n-i}(X,\mathcal{W} \Omega_X^i)/VH^{n-i}(X, \mathcal{W}\Omega_X^i)$
	\item $a'_i = \dim H^{n-i}(X, \mathcal{W}\Omega_X^i)/F H^{n-i}(X,\mathcal{W}\Omega_X^i)$
\end{itemize}
then 
\[
h^0 = a_0\ \ h^i=a'_{i-1} + a_i  (1 \leq i \leq n) \ \ a'_n=0
\]
and 
\[
\begin{aligned}
&\text{Hdg}(t) = \begin{cases} (0,0) & t=0 \\
(\sum_{i \leq t}h^i, \sum_{i\leq t} i h^i)& 1 \leq t \leq n+1\\
\end{cases}\\
&\text{Nwt}(t) = \begin{cases}
(0,0)& t=0\\
\text{Hdg}(t) +(a'_{t-1}, ta'_{t-1})& 1 \leq t \leq n+1\\
\end{cases}
\end{aligned}
\]
\begin{seccor}
	If $H^j(X, \mathcal{W}\Omega_X^i)$ are all torsion-free, then $X$ being ordinary is equivalent to $\text{Nwt}(t) = \text{Hdg}(t)$.
\end{seccor}
\end{document}
